\documentclass[11pt]{article}
\usepackage[utf8]{inputenc}
\usepackage{amsmath,amssymb}
\usepackage{graphicx}
\graphicspath{ {./images/} }
\usepackage[margin = 1.2 in]{geometry}
\usepackage{titlesec}
\usepackage{textcomp}
\usepackage{sidecap}
\usepackage{verbatimbox}
\sidecaptionvpos{figure}{t}
\usepackage{wrapfig}
\usepackage[margin=0cm]{caption}
\captionsetup[figure]{labelfont={bf},name={Figure},font=footnotesize}
\usepackage{multicol}

\begin{document}
\begin{titlepage}
    \title{\vspace{40pt}\Huge KiriZen: Creation of an iPhone application in swift based on the art of kirigami}

    \author{ \\\hline \\\\\\ \Large \vspace{5pt} Author: Stephanie Tindal \\\Large \vspace{5pt}{Student ID: 1936508}\\\Large \vspace{50pt}{Supervisor: Achim Jung} \\ \hline\\\\\\  \vspace{5pt}{MSc Computer Science} \\ \vspace{5pt}{School of Computer Science, University of Birmingham}\\\\\\ \hline\\\\}

    \date{September 2019 \vspace{20pt}}
    \maketitle
\end{titlepage}
\begin{frame}{}
    \begin{multicols}{2}
    \tableofcontents

\end{multicols}
\end{frame}

\newpage

\begin{abstract}
    Abstract abstract 

\end{abstract}

\vspace{50pt}
\renewcommand{\abstractname}{Acknowledgements}
\begin{abstract}
    I would like to thank Professor Achim Jung for supervising me throughout the project and Dr Iain Styles for the initial idea of an origami based application. 
\end{abstract}

\newpage
\section{Introduction}

        \subsection{A Brief History of Origami and Kirigami}
        
            \paragraph{}
            Origami is the art of folding paper. One of its earliest depictions was the folding of an ancient Egyptian map made from papyrus. Upon the invention of paper, paper folding was used mainly in religious ceremonies due its high price. There are examples of paper folding across many cultures therefore it is difficult to say where and when it was invented. In japan paper folding was popularised 
            % \cite{origami bible stuff}
            kirigami
            in Japanese, "kiri" means "to cut"
            \cite{Temko2004kirigami}
          
            Kirigami in addition to origami includes cutting the paper. 2 
            Houdini 
    
            \subsubsection{Relevance}
                \paragraph{} 
                Although paper folding used to be used in mostly religious contexts, origami has become more recreational in nature. 
            
                \paragraph{} 
                Origami and kirigami are now seen in pop-culture and art galleries. 
                %star wars
                %fashion
                %articles
                %books
                I learnt origami and kirigami from Chinese and Japanese students my grandmother hosted when I was younger and remember searching for books in the UK but not being able to find them in any libraries in my hometown. Now these types of books are widely available in most bookshops and libraries sometimes which entire sections dedicated to them. This increase in popularity or origami and kirigami.
            
                In addition to this,  origami and kirigami 
            
            
                % \cite{origami bible stuff}
    
                In recent years
            
                relaxation and mindfulness activities and applictions growing in popularity and including all ages such as adult colouring books, origami packs, head space applicatoin, 
                
        
        \subsection{Idea Conception}
            
                    \paragraph{} 
                        Having been a keen creator of origami from the age of 4, it was suggested by one of my professors that I create an application that involves origami. Upon discussion with my supervisor we decided the focus should be on kirigami - a branch of origami which involves cutting as well as folding the paper as this can be a 2D application. The decision was made to have two aspects of the app - creation and trying to create a given pattern. As discussed above, the popularity of kirigami is growing as is the popularity of relaxing head space apps. I only found a handful of apps - discussed in the background research later on - focused on kirigami on the app store therefore I believe there is a place for the app I have created and the market is not over saturated. 
                        
                        \subsection{Swift}
               
                \paragraph{} 
                    I decided to create an iPhone application as this was an area of coding I had not yet explored and was keen for the challenge of a new language. To create an iPhone application, Swift is a common language to use with some Objective-C still being used on occasion.  
                    Xcode is the integrated development environment (IDE) used for Swift.
        
                \subsection{Project Aim}
            
                 \paragraph{} 
                 %                  The aim of the project is to create an application in which users can create cuts in a virtually folded sheet of paper and the application will reveal the unfolded pattern to them upon the click of a button. The application will 
                 
                 %                  The next level will allow the users to try and imagine how to cut the shape that is shown from the folded piece of paper.
                 
                 % The application will have to be virtually appealing and


                % • Relevance: The app will allow users to train their 3D imagination by interacting with the application. 
                
                % In addition to the application itself, users can mimic the patterns created in real life increasing their creativity.
                
                %                  1. Learn the basics for creating an iPhone application
                % 2. Create an application for iPhone that allows the user to draw cuts into a folded piece of
                % paper
                % 2.1. Demonstrate a crease pattern and folding of a square piece of paper into 2
                % 2.2. Allow the user to draw cuts into the piece of paper
                % 2.3. Reveal the image of the paper unfolded with the cuts made
                % 2.4. Repeat for folding the square into 4, 6 (as you cannot fold a piece of paper into 3 or
                % 5 layers about a central point) – revealing the unfolded cut versions at the end.
                % 3. Add another level to the application which has an unfolded target shape and asks you what cuts are to be made to create this shape. E.g. pentagon is the target shape. The
                % application reveals the square folded into 5 and asks you where the cut should be made. 3.1. A margin of error will have to be programmed into this.
                % 3.2. Application will show you the solution once you have tried.
                % 4. Additional features
                % 4.1. Additional features can be programmed for example changing the colours and
                % shading.
                % 4.2. Higher number of folds.
                % 4.3. Showing the actual folding/unfolding of the piece of paper in between steps
                 

\newpage
\section{Background Research}

        \subsection{Kirigami-like Applications}
           \paragraph{}
            I searched for applications on the web and iPhone app store with a theme that centered around creating kirigami. I wanted to gain an insight into what is already out there to make sure my idea was unique, and to evaluate the strengths and weakness of the other apps as a user and learn from them before I design my own app. 
        
            \subsubsection{Paper Snowflake Maker}
            
                \paragraph{Type:} Web application %\cite http://rectangleworld.com/PaperSnowflake/
                
                \paragraph{Description:}
                Paper Snowflake Maker is a website where the user can cut out polygons from the shape of a ready folded piece of paper by clicking on the screen to make the corners of the polygon (right panel in \textbf{Figure~\ref{fig:paperSnowflakeMaker}}). The paper snowflake is virtually created and revealed to the user (left panel in \textbf{Figure~\ref{fig:paperSnowflakeMaker}}).
                    \begin{figure}[ht]\centering\includegraphics[width=0.9\textwidth]{Images/paperSnowflakeMaker}
                        \caption{
                        \label{fig:paperSnowflakeMaker}
                        A full page screenshot taken from the web application Paper Snowflake Maker. The user has made cuts on the folded image and is about to complete their next cut. The green dot indicates the beginning of a cut, the red dots indicate the corners of the polygon to be cut. When the users mouse passes over the green dot - an additional circle is created around its edge indicating to the user that if they click there, the polygon they created will be cut from the virtual piece of paper (the latter scenario is illustrated in this figure). The corners of the polygon can be dragged to a different shape before the cut is finalised.}
                    \end{figure}
                    
                \paragraph{Strengths:}
                It is clear how to make cuts in the virtual folded piece of paper as soon as you start clicking on it, as seen on the right panel in \textbf{Figure~\ref{fig:paperSnowflakeMaker}}. The green and red dots that appear as you click are clear indicators for the progression of the cut.
                
                The auto update button is a convenient feature as the user does not have to click the "make snowflake!" button every time a new shape is cut. 
                
                The draggable corners are an additional convenient feature if the user would like to edit the shape of the polygon before the cut is created.
                 
                 The shading of the final snowflake provides a sense of realness as its shadows appear as a folded pieces of paper would.
                 
                \paragraph{Weaknesses:}
                When using the application, the add button (which finalises the cut - mostly meaning the construction lines on the folded paper image are removed) seems unnecessary as it has the same functionality as other actions, e.g. starting to draw a new shape, or clicking the "make snowflake!" button has the same result.
                
                The cuts that can be made on this virtual snowflake are much more complex than those which can recreated on a real piece of paper, this slightly takes away from the sense of realism indicated by the accurate shadows of the folds. The smaller pieces that are created in \textbf{Figure~\ref{fig:paperSnowflakeMaker}} would fall off after being severed from the main larger component also contrasting the realism.
                
                The save image to your computer button opens another window explaining how to save the image (using "Save Image As...") which is unnecessary and ironically the user is not able you to save the image on their computer via this pop-up window. Copying the image produced a low resolution image as does saving the image straight from the original screen. In addition the "share image on Facebook button does not work."
                
                \paragraph{Lessons Learned:}
                The cutting is intuitive for this web application - I can consider this approach when developing my own cutting method. The overall look of the application is clean, however the buttons on the side and instructions could be updated for a more modern feel and possibly rearranged. Buttons that do not provided the intended functionality should be discarded (such as the save image button).
                
            \subsubsection{Snowflake!}
            
                \paragraph{Type:} iPhone application 
            
                \paragraph{Description:}
                Snowflake is an application that allows the user to cut out free-form shapes from either 22.5\textdegree{} segments or 45\textdegree{} segments. The image revealed is a snowflake-like shape as shown in \textbf{Figure~\ref{fig:snowflakeResult}} which is created from the 22.5\textdegree{} segments in \textbf{Figure~\ref{fig:snowflakeSegmentCut}}. 
                    
                    \begin{figure}[!ht]
                        \begin{minipage}{0.45\textwidth}
                            \centering \includegraphics[width=0.7\linewidth]{Images/snowflakeSegmentCut}
                            \caption{A ready made cut resulting from the "Random" button. The segment size selected is 22.5\textdegree{}. Pressing the "View Snowflake" button takes the user to the screen in \textbf{Figure~\ref{fig:snowflakeResult}}.\\}
                            \label{fig:snowflakeSegmentCut}
                        \end{minipage}\hfill
                        \begin{minipage}{0.45\textwidth}
                            \centering
                            \includegraphics[width=0.7\linewidth]{Images/snowflakeResult}
                            \caption{The result of the cut made in \textbf{Figure~\ref{fig:snowflakeSegmentCut}}. The colour of the image can be altered with the colour pickers below it. The image can be shared with the buttons in the "Share" section providing the relevant button works.}
                            \label{fig:snowflakeResult}
                        \end{minipage}
                    \end{figure}
                    
                    
                    
                \paragraph{Strengths:}
                A version of Tchaikovsky's Dance of the Sugar Plum Fairy plays in the background which is synonymous with Christmas in keeping with the theme of the application. 
                The user can use the random shape button to create a base to start with. The user can change the colour scheme of the snowflake if desired.
                
                \paragraph{Weaknesses:}
                Snowflakes have six sides due to the shape of the H\textsubscript{2}O molecule. Eight sided crystals would never be formed naturally therefore it is an odd choice for this application to choose pieces of paper folded into segments with angles of 22.5\textdegree{} and 45\textdegree{}, as this forms eight or sixteen sided snowflakes rather than the more common six or at most twelve sided snowflakes. %\cite{http://www.its.caltech.edu/~atomic/snowcrystals/unusual/unusual.htm} 
                
                The area of the screen dedicated to the main function of creating cuts from a segment of the final shape consists of a very small portion of the screen. The ratio is especially low when cutting from the 22.5\textdegree{} segments as seen in \textbf{Figure~\ref{fig:snowflakeSegmentCut}}.
                
                The shape that is cut from the shape drawn is not always as expected. For example in \textbf{Figure~\ref{fig:snowflakeSegmentCut}} this does not reflect the cut you would expect from the free-form shape drawn by the user in \textbf{Figure~\ref{fig:snowflakeOutline}}.
               
                On the share bar seen in the centre of \textbf{Figure~\ref{fig:snowflakeResult}}, the "Send to a friend!" button has exactly the same response as the "E-Mail" button. It is also slightly  misleading as I would expect this button to create a message via WhatsApp or iMessage rather than email if sending to a friend. The upload icon button allows the user to create a "Happy new year" themed postcard and when the "Send" button is pressed, the pop-up disappears leaving the user with no idea where the image they just created was saved or sent unlike with the buttons that do work - the user is sent to the relevant email or social media screen. The same situation occurs when the twitter button (icon of a bird) is pressed - no response occurs after the "Send" button is pressed.
                
                The status bar disappears on the app so the user is not able to see information such as signal, battery life time etc while using the application. 
                 
                 \begin{figure}[!ht]
                        \begin{minipage}{0.45\textwidth}
                            \centering \includegraphics[width=0.7\linewidth]{Images/snowflakeOutline}
                            \caption{A free-form line is drawn on a 45\textdegree{} segment to indicate where the user has drawn the cut.\\\\}
                            \label{fig:snowflakeOutline}
                        \end{minipage}\hfill
                        \begin{minipage}{0.45\textwidth}
                            \centering
                            \includegraphics[width=0.7\linewidth]{Images/snowflakeCut}
                            \caption{The result of the free-form line drawn in \textbf{Figure~\ref{fig:snowflakeOutline}}. There is a bug in the application as this is not the cut the user would expect from what they have drawn.}
                            \label{fig:snowflakeCut}
                        \end{minipage}
                    \end{figure}
        
                \paragraph{Lessons Learned:}
                An application should be researched before it is made and assumptions checked - such as the assumption that was made about the dimensions of a snowflake. If the functionality of the buttons in the app is not working, there should be a response message from the application to inform the user as to what the problem is. Each button should have unique functionality. The cuts that are to be made in the virtual piece of paper should take up a large portion of the screen as this is the main area the user is interacting with.
            
            
            \subsubsection{Kirie}
            
                \paragraph{Type:} iPhone application 

                \paragraph{Description:}
                This application allows the user to draw lines (which are transformed into cuts) on a virtual piece of paper folded in half vertically. The virtual piece of paper is unfolded upon reveal. 

                \paragraph{Strengths:}
                The majority of the screen is used when creating the cut. This allows you to make intricate cuts and the overall feel of the app is minimalist and clean. 
                
                The option to load an image from the users library to cut from is available to either cut from, or to use as a background image behind the cuts.
                
                 \begin{wrapfigure}{r}{0.25\textwidth}
                    \centering
                    \includegraphics[width=0.25\textwidth]{Images/kirieMain.PNG}
                    \captionsetup{{margin = 0.2cm}}
                    \caption{The home screen of the kirie app. The three buttons at the bottom of the scree do not provide their functionality or cause the app to crash. Only the "Cut" button has the expected action of taking the user to the next screen.}
                    \label{fig:kirieMain}
                \end{wrapfigure}
                
                
                \paragraph{Weaknesses:}
                On the in initial screen (seen in \textbf{Figure~\ref{fig:kirieMain}}), only the "Cut" button works. It is also not very clear that the "Cut" button is a button as could be  mistaken for a logo especially as the other buttons of the screen are of a different format. There is also no highlight or shading response for the button which in this case when the reaction to the button is slow, can indicate to the user that it is in deed not a button as there is no response shown within a regular period of time. If the "download" button is pressed, the app crashes and closes. If the "How to" button is pressed, there is no response from the app. If the "user" button is pressed, a pop up shows ups prompting a username. With any input the application closes the popup and returns to the same screen as before - there also appears no way to sign up or login. It should also be noted that the captions for the buttons do not have consistent capital letters to start ("How to" does, the other buttons do not), a minor flaw that shows lack of attention to detail.
                
                On the finished cut screen shown in \textbf{Figure~\ref{fig:kirieCut}} the save icon button does not save the image anywhere despite a conformation pop-up. It is unclear what the functionality of the picture icon button is as the cuts have already been finalised. The upload icon results in a conformation pop-up (saying the image has been uploaded) but no upload has actually occurred - the user was not even asked where they would like the image to be uploaded. 
                
                The trace from the users finger is very sensitive and can result in cuts that are not smooth. The same shape cut e.g. the top and bottom "X" created in \textbf{Figure~\ref{fig:kirieXs}} results in two cuts seen in \textbf{Figure~\ref{fig:kirieCut}} that are a) not consistent with cuts that would occur with a cutting knife or scissors therefore it is difficult for the user to predict what shaped of cuts will occur from the lines they draw, b) similar lines result in different cuts due to the starting point of each line.
                
                \begin{figure}[!ht]
                        \begin{minipage}{0.45\textwidth}
                            \centering \includegraphics[width=0.7\linewidth]{Images/kirieXs.png}
                            \caption{The lines drawn by the user on the left panel. The top "X" although it looks very similar to the bottom "X" produces a different cut as shown in  \textbf{Figure~\ref{fig:kirieCut}}.\\\\}
                            \label{fig:kirieXs}
                        \end{minipage}\hfill
                        \begin{minipage}{0.45\textwidth}
                            \centering
                            \includegraphics[width=0.7\linewidth]{Images/kirieCut.png}
                            \caption{The result of the "X"s drawn in \textbf{Figure~\ref{fig:kirieXs}}. The right hand side is a reflection of the left. The cuts are not what would be expected from the "X" lines drawn and the results for each X would be more similar rather than a reflected shape.}
                            \label{fig:kirieCut}
                        \end{minipage}
                    \end{figure}
                
                \paragraph{Lessons Learned:}   
                A big portion of the screen allows the user to have more control and allows a greater space for creativity.
                
                It is useful to have all the buttons of a similar format for the application so they can all be recognised and used correctly.
                
                The cuts created do not have to be closed shapes resulting in unexpected shapes being created after the user confirms the cut. The application should restrict the user to only creating closed shapes as that would allow the user to predict where the cuts would be made resulting from their line. When the cuts are mirrored, the resolution of the image of the cuts decreases.
               
                If the functionality of the buttons causes the application to crash, this is a major error and is unacceptable for any application. All buttons should be thoroughly tested throughout creation of my application. There could be conformation pop-ups in my application but only when the functionality works unlike in "kirie". 
                
                Overall the usefulness of the app is questionable as it just reflects the cuts in the y-axis and there is no other refection or folding option. Additional features that add little to the app and are not completely necessary or relevant should not be implemented. Any non-core features should be added after the main functionality of the application works well. The variety of tasks to do in the app is very limited.
                
                
                 \subsubsection{Handcraft Snowflakes}
            
                \paragraph{Type:} iPhone application 

                \paragraph{Description:}
                This application uses the character of Peppa Pig throughout. Peppa describes comments on the application which consists of cutting shapes out of a segment of a snowflake. There is gallery available where your snowflakes can be saved. Coins can be collected while using the app or can be purchased. They are redeemed for a new ready made shapes to cut out the snowflake, new background textures/colours for the snowflake paper, or to add a pages in the gallery booklet for saving more snowflakes. 

                \paragraph{Strengths:}
                The theme is consistent through the use of a character. Some of the audience for Peppa pig may be directed to this application on the app store. 
                
                There are many language options so the character of Pepper Pig speaks in the language selected.
                
                There is a large proportion of the screen dedicated to cutting shapes from the snowflake. In this case the snowflake has a reasonable number of sides. 

                \begin{wrapfigure}{r}{0.25\textwidth}
                    \includegraphics[width=0.25\textwidth]{Images/peppa/peppaRussian.png}
                    \captionsetup{{margin = 0.2cm}}
                    \caption{The current language selected is Spanish, however the colour picker has a title in Russian.}
                    \label{fig:peppaRussian}
                \end{wrapfigure}
                \paragraph{Weaknesses:}
                
                The commentary by Peppa pig does not contain much helpful information on how to use the app, it merely states the obvious such as "Lets start making a snowflake". The characters name is not in the title for the application, this means there is less exposure to the Peppa pig audience.
                
                After selection of a language, the language is inconsistent throughout the application. For example no matter which language you pick, the title for some sections - for example the colour picker - is always displayed in Russian as seen in \textbf{Figure~\ref{fig:peppaRussian}}.  As the application is clearly aimed for children this may be confusing for them.3
                
                There is a black loading page before every screen is displayed which stops the flow of the app and makes the app feel slow.
                
                There are glitches on the free-form creation screen displayed in \textbf{Figure~\ref{fig:peppaGlitchLine}} - you can still make cuts on the paper via the underlying view when a pop up is displayed. This should be disabled and the line should be displayed in the layer below the pop up.
                

                
                \begin{figure}[!ht]
                        \begin{minipage}{0.32\textwidth}
                            \centering \includegraphics[width=0.8\linewidth]{Images/peppa/peppaGlitchLine.PNG}
                            \captionsetup{{margin = 0.2cm}}
                            \caption{The user can still interact with the line on the screen when this can make the action they wish to carry out ambiguous as the exit screen pop-up is displayed.}
                            \label{fig:peppaGlitchLine}
                        \end{minipage}
                        \begin{minipage}{0.32\textwidth}
                            \centering
                            \includegraphics[width=0.8\linewidth]{Images/peppa/peppaMain.PNG}
                             \captionsetup{{margin = 0.2cm}}
                             \caption{The main screen where users can select the coloured paper to cut snowflakes or the notepad to view their snowflake gallery.\\}
                            \label{fig:peppaMain}
                        \end{minipage}
                        \begin{minipage}{0.32\textwidth}
                            \centering
                            \includegraphics[width=0.8\linewidth]{Images/peppa/peppaGlitchPoint.PNG}
                             \captionsetup{{margin = 0.2cm}}
                             \caption{A hand appears on screen prompting the user to interact with the phone or shoes. This is misleading the user cannot interact with any object in the current frame.}
                            \label{fig:peppaGlitchPoint}
                        \end{minipage}
                    \end{figure}
                    
                    It is unclear what objects can be interacted with throughout the application for example in \textbf{Figure~\ref{fig:peppaMain}} the scissors would be the most obvious object to interact with to start cutting the snowflakes, however the user can only interact with the coloured paper and the notepad. On occasion there appears to be a glitch illustrated in \textbf{Figure~\ref{fig:peppaGlitchPoint}} where the application indicates to the user an item should be tapped, however in fact none of the objects in the frame are intractable. 
                    
                     The app will freeze if your cut touches itself as seen in \textbf{Figure~\ref{fig:peppaFreeFormCut}}. Sometimes the paper is not cut even when the line the user is drawing is registered. If the user moves their finger too quickly, the cut may not be created and the registered line will disappear as if illegal cut was drawn.
                    
                    \begin{figure}[!ht]
                        \begin{minipage}{0.32\textwidth}
                            \centering \includegraphics[width=0.8\linewidth]{Images/peppa/peppaShapes.PNG}
                            \captionsetup{{margin = 0.2cm}}
                            \caption{The user can drag and drop shapes onto the paper, resizing the shapes if desired. There is the option to pick more shapes if the button with multiple shapes is pressed.}
                            \label{fig:peppaShapes}
                        \end{minipage}
                        \begin{minipage}{0.32\textwidth}
                            \centering
                            \includegraphics[width=0.8\linewidth]{Images/peppa/peppaFreeFormCut.PNG}
                             \captionsetup{{margin = 0.2cm}}
                             \caption{The user is able to create free-form cuts with their finger on this screen. Many glitches occur with the free-form cut creation page of the app.//}
                            \label{fig:peppaFreeFormCut}
                        \end{minipage}
                        \begin{minipage}{0.32\textwidth}
                            \centering
                            \includegraphics[width=0.8\linewidth]{Images/peppa/peppaPopUp.png}
                             \captionsetup{{margin = 0.2cm}}
                             \caption{This pop-up is usually displayed if a button that will exit the creation screen is pressed. It is unclear if the tick means to keep the shape or get rid of it.}
                            \label{fig:peppaPopUp}
                        \end{minipage}
                    \end{figure}
                    
                
                    On the ready made shapes page (\textbf{Figure~\ref{fig:peppaShapes}}), it is not obvious that you need to drag the shapes of the cuts as they appear where the buttons are and are covered by the button bar when they appear and they only appear if you drag your finger not when you tap the button. The button will register the tap by flashing but no reaction will occur from the app if the user only tap not drag your finger.  even though the buttons glow when tapped. The ready made cuts produce a jagged edge if the user enlarges them as seen in cut made in the top left of \textbf{Figure~\ref{fig:peppaShapes}}. Some of the ready made cuts are not in keeping with the theme such as duck and fish shapes for a snowflake / Christmas application. 
                
                The main problem with the application is the assumed knowledge of the buttons. For example, it is unclear what the buttons with snowflakes on them in \textbf{Figure~\ref{fig:peppaMain}} and \textbf{Figure~\ref{fig:peppaGlitchPoint}} will do. They appear to allow the user to drag snowflakes onto the window in the background, however it is difficult to use so the functionality is unclear. The left facing arrow button on the main screen shown in \textbf{Figure~\ref{fig:peppaMain}} will take the user back to the previous screen. This navigation to a previous screen is a common action for the left arrow button. However on the free-form cut screen shown in \textbf{Figure~\ref{fig:peppaFreeFormCut}}, pressing the left arrow button will undo the shape the user has just cut.
                 On shape cutting screen, there is no undo option for cuts made. There is lack of consistency with the image of the button making use of the application unnecessarily complex.
                
                The adverts are very intrusive to the user experience and even when on mute, the animation of Peppa pig still continues to animate speaking. The user cannot skip this section and the animation still continues to speak when the mute button within the application or on the phone itself is activated. A pop up (\textbf{Figure~\ref{fig:peppaPopUp}}) occurs if a button is pressed which will exit the user from the cutting screen, however it is unclear which button to press to continue working on the shape or to discard it and the pop-up occurs very frequently while using the application as multiple buttons activate it which is also disruption. 
                
                The name of the application is too long for the icon, therefore you cannot see the entire name of the app on iPhone screen only "Handcraft:S...".
                
                \paragraph{Lessons Learned:}   
                Additional features such as sound, language and animation should not take away from the application experience, only enhance it and only when the core features work as this is more important. Using images for buttons is tricky, it is done well in some cases, e.g. fig x but not in others, e.g fig y. The buttons should have clear functionality and clear if the images are buttons that can be interacted with or not - unlike in fig z. 
                
        \subsection{Cutting Paper Applications}
            \subsubsection{PaperCutCraft}
            
                \paragraph{Type:} iPhone application 
                 
                \paragraph{Description:} This application is for cutting paper

                \paragraph{Strengths:}
                There is a large amount of progression that can be had in the application with opportunity for
                
                \paragraph{Weaknesses:}
                While using the app, on occasion the cutting action stops working. The app creates a line representing a cut as usual, but it does not result in the paper being cut. After this occurs, there is no way to progress in the app even if the refresh button is used. There is another glitch where only the thin line drawn (e,g, figure a) is cut from the screen rather than the entire section of paper the user would expect to be cut from their trace as before. 
                It is unclear how to complete a level, the initial idea appear to cut but in order to get a score, you must press the forward triangle button "this is usually associated with a "play" feature which can be confusing to the user. 
                The shadow illusion can also be "cut off" breaking the illusion of the point of the shadow. 
                
                It is actually quite difficult to trace a thin line on this app as their finger is in the way
                
                The triangle button - the center button seen in fig a and b - means both finish the cut you are on, and move when the sa seen in figure c.  
                
                It is difficult to see what criteria there is for the accuracy rating as you use the aplication. 

                
                \paragraph{Lessons Learned:}
                
                \subsubsection{Topetope}
                 
                \paragraph{Type:} iPhone application 

                \paragraph{Description:}
                "TOPETOPE - cut the square" is an iPhone application that is advertised as a game where the user can cut a square piece of paper and there are cutting challenges that may come with time limits. The application did not appear to be working when tested multiple times with every effort made to troubleshoot it. Most of the application appeared to be inaccessible. 
                
                \paragraph{Strengths:}
                When the application does work (the initial screen shown in \textbf{Figure~\ref{fig:topeReality}} was somewhat functional), the cut of the paper is seamless. The user can drag their finger from one corner to another and the paper is cut along the line and one section falls off via an animation.
                
                \begin{figure}[!ht]
                        \begin{minipage}{0.45\textwidth}
                            \centering
                            \includegraphics[width=0.6\linewidth]{Images/topeReality.png}
                            \captionsetup{margin = 0.5cm}
                            \caption{The initial (and only accessible) screen for the app Topetope. The user can use their finger to cut the piece of paper. The application responds approximately half the time. }
                            \label{fig:topeReality}
                        \end{minipage}
                        \begin{minipage}{0.48\textwidth}
                            \centering
                            \includegraphics[width=0.85\linewidth]{Images/topeAdvertising.jpg}
                            \captionsetup{margin = 0.5cm}
                            \caption{The description for Topetope on the iPhone App Store.}
                            \label{fig:topeAdvertising}
                        \end{minipage}
                    \end{figure}
                    

                \paragraph{Weaknesses:}
                The application is extremely different to the advertised app as seen in figure \textbf{Figure~\ref{fig:topeAdvertising}}. In reality the user is only able to access one screen. With the level that can be accessed, it is unclear what the goal is.
                
                There is no home screen option, navigation available or the presence of tutorial page or directions. The cutting of the paper with the users finger only works approximately half the time.
                
                The glitching and lack of response from the app is very frustrating.
                
                
                There is no clear meaning of the word "topetope". May be translated from Spanish meaning the top or stop, however any translation of it does not particularly make sense in context. 
                
                 Topetope also lacked consistency as its description was different to the application.

                \paragraph{Lessons Learned:}
                In contrast to Topetope, the application I create should have a clear and obvious goal. A tutorial may be necessary as this would have been useful in this case. My application should be thoroughly tested at each stage and needs to reliably work throughout. This is especially important due to the relaxing mindfulness component of my application. Another aspect to consider is if the user is cutting the paper, with Topetope is not clear which side of the paper will be disappearing after it is cut. I need to make this decision clear if I am making it for the user, or allow the user to choose for themselves. 


       \subsection{Take home from reviewing other applications}
       
            \paragraph{}
            Searching for web applications as well of iPhone apps turned out to be a useful path as Paper Snowflake Maker was the only application where the cutting action worked reliably.
            
            All of the iPhone applications I tested had inconsistencies with their main functionality of cutting shapes out of the segments / virtual paper. However the web application Paper Snowflake Maker did not have such an inconsistency. Whatever shape was drawn, the expected cut is created when the cut is finalised and when the unfolded virtual paper is revealed. The application I create should be consistent, reliable and the main feature of cutting should be easy to use.  
            
            Many of the examples had features which were either did not fully carry out the expected function or provided no response when interacted with. I will place importance on making sure every feature available work well and to the intended purpose over implementing many features. 
            
            None of the iPhone applications I looked at had a cut action that was consistent and worked reliably. All had some sort of glitch to do with the cut. The applications with a larger work space were nicer to use. 
            
            A few of the application had buttons where their intended functionality ambiguous, some where the same button actually changed functionality depending on the screen. If I choose to use icons, the action of the button should be apparent. Consistency with the format of the buttons can also be important e.g. the buttons I create could all have a similar style for navigation. Some of the applications above lacked this consistency with their buttons. 
            
            Many of the applications reviewed had a snowflake / Christmas theme. Although this may be a good marketing tactic, I do not wish to restrict the kirigami to 3 folds (6 segments) and would like to make something diverse that is different to what is already available and not restricted to seasonal use. I also do not want to restrict the age range by creating a character like pepper pig. However I will try and extract the features that kept a coherent theme in that application.
            
            The size of the virtual paper should be large enough on the screen to allow for ease of interaction. The method which the user interacts with the paper should be simple and where the cut will occur should be clear (e.g. clear which part of the paper is cut - an inverted cut should be easy to carry out). 
            With a few of the application, another major flaw was that there is no clear goal or task. I should try to create an application where the goal or task is obvious.


\newpage
\section{Prototype}

    \subsection{Requirements}

        \paragraph{}

    \subsection{Target User}
    
            \paragraph{}
            Similar to the target audience for origami and kirigami
            no specific age
            
            outline applicable to each age
            kids
            young professionals 
            elderly generation 
            Application can be interacted in a short period of time, ideal for commuting. Mindfulness 
            

    \subsection{Wire Frames}
    Initially I created sketches of screen for the first generation prototype and used the application "POP - Prototyping on Paper" to make the sketches interactive and test it out on potential users. The users can click on buttons to link them  to the next screen as seen in fig x %cite
    
        POP images

    Feedback was mainly positive, especially as wire frames are not commonly interactive, this allowed for a more authentic experience for the user. The diagrams were clear so it was easy for the user to understand what was going on and the concept of the app even with just links to the different pages as the interactivity.
    More relevant titles for pages or no titles for some pages. Home button on every page is unnecessary as the app does not have a large number of screens. Share button could potentially be included. Undo and clear buttons should be included when creating the cuts. 
    
    
    After collating the feedback, I translated the sketches to generation two prototypes on balsamiq which I was familiar with as I used them on projects throughout the year. %https://balsamiq.com
    I kept in mind the requirements while creating the screens for the application.
    
\clearpage
             \begin{figure}
                \begin{minipage}[c]{0.35\textwidth}
                \includegraphics[width=1\textwidth]{Images/Prototype/prototypeHomeScreen.png}
                \end{minipage}\hfill
                \begin{minipage}[c]{0.65\textwidth}
                \captionsetup{font={normalsize}, margin = 1cm}
                \caption{The home screen will display the graphics for the two parts of the application - "Create Pattern" and "Match Pattern". The images used for the buttons will display a pattern being created and a target image respectively. Clicking on the "Create Pattern" button will link the user to \textbf{Figure~\ref{fig:chooseFold}}, and clicking on match pattern will take the user to  \textbf{Figure~\ref{fig:target}}}
                \label{fig:homeScreen}
                \end{minipage}
            \end{figure}
            \begin{figure}
                \begin{minipage}[c]{0.65\textwidth}
                \captionsetup{font={normalsize}, margin = 1cm}
                \caption{This figure is where the user will be able to choose how they want their virtual piece of paper folded. The images for each button will show the virtual piece of paper folded in half, quarters, sixths and eights with straight edges. You cannot fold a piece of paper into odd numbers about a centre point which is why they are excluded here. The user will press one of these buttons which will link them to the corresponding version of the screen shown in \textbf{Figure~\ref{fig:createPattern}}}
                \label{fig:chooseFold}
                \end{minipage}\hfill
                \begin{minipage}[c]{0.35\textwidth}
                \includegraphics[width=1\textwidth]{Images/Prototype/prototypeChooseFold.png}
                \end{minipage}
            \end{figure}
            
            \begin{figure}
                \begin{minipage}[c]{0.35\textwidth}
                \includegraphics[width=1\textwidth]{Images/Prototype/prototypeCreatePattern.png}
                \end{minipage}\hfill
                \begin{minipage}[c]{0.65\textwidth}
                \captionsetup{font={normalsize}, margin = 1cm}
                \caption{The user will create the cuts from the virtually folded piece of paper on this screen. The shape of the folded segment displayed will depend on which button was pressed.
                They will have the option of deleting the last cut created with the "Undo" button, or refreshing the canvas with the "Clear" button.}
                \label{fig:createPattern}
                \end{minipage}
            \end{figure}
            
            \begin{figure}
                \begin{minipage}[c]{0.65\textwidth}
                \captionsetup{font={normalsize}, margin = 1cm}
                \caption{Caption}
                \label{fig:reveal}
                \end{minipage}\hfill
                \begin{minipage}[c]{0.35\textwidth}
                \includegraphics[width=1\textwidth]{Images/Prototype/prototypeReveal.png}
                \end{minipage}
            \end{figure}
            
           \begin{figure}
                \begin{minipage}[c]{0.35\textwidth}
                \includegraphics[width=1\textwidth]{Images/Prototype/prototypeTarget.png}
                \end{minipage}\hfill
                \begin{minipage}[c]{0.65\textwidth}
                \captionsetup{font={normalsize}, margin = 1cm}
                \caption{Caption}
                \label{fig:target}
                \end{minipage}
            \end{figure}
                           
           \begin{figure}
                \begin{minipage}[c]{0.65\textwidth}
                \captionsetup{font={normalsize}, margin = 1cm}
                \caption{Caption}
                \label{fig:matchPatternCreate}
                \end{minipage}\hfill
                \begin{minipage}[c]{0.35\textwidth}
                \includegraphics[width=1\textwidth]{Images/Prototype/prototypeMatchPatternCreate.png}
                \end{minipage}
            \end{figure}
                        \clearpage

            \begin{figure}
                \begin{minipage}[c]{0.35\textwidth}
                \includegraphics[width=1\textwidth]{Images/Prototype/prototypeMatchPatternCreateHelp.png}
                \end{minipage}\hfill
                \begin{minipage}[c]{0.65\textwidth}
                \captionsetup{font={normalsize}, margin = 1cm}
                \caption{Caption}
                \label{fig:matchPatternCreateHelp}
                \end{minipage}
            \end{figure}
            
            \begin{figure}
                \begin{minipage}[c]{0.65\textwidth}
                \captionsetup{font={normalsize}, margin = 1cm}
                \caption{Caption}
                \label{fig:revealWithOverlay}
                \end{minipage}\hfill
                \begin{minipage}[c]{0.35\textwidth}
                \includegraphics[width=1\textwidth]{Images/Prototype/prototypeRevealWithOverlay.png}
                \end{minipage}
            \end{figure}
                        \clearpage
            
            
                \paragraph{}
                Covergirl Covergirl
            

\newpage
\section{Creation of Software}

    
    \subsection{Creating Kirigami}
            \paragraph{}
            \subsubsection{Choose Fold}
                \begin{wrapfigure}{r}{0.25\textwidth}
                        \centering
                        \includegraphics[width=0.25\textwidth]{KiriZen/chooseFold.png}
                        \caption{Icon}
                        \label{fig:kiriZen-chooseFold}
                    \end{wrapfigure}
                    
            Main screen
            
            Music and Info discussed later on in additional features (x.x.x.)
            
          
            \paragraph{Segue}
                I used the storyboard in Xcode to lay out the general overview of the application. This visualisation was handy when creating segues between the classes as I could see clearly which classes were interacting. 
                
            \subsection{Create Pattern}
             \begin{wrapfigure}{l}{0.25\textwidth}
                        \centering
                        \includegraphics[width=0.25\textwidth]{KiriZen/createPattern.png}
                        \caption{Icon}
                        \label{fig:kiriZen-createPattern}
                    \end{wrapfigure}
            \paragraph{Canvas}

            % \begin{wrapfigure}{r}{0.25\textwidth}
            %             \centering
            %             \includegraphics[width=0.25\textwidth]{KiriZen/icon.png}
            %             \caption{Icon}
            %             \label{fig:icon}
            %         \end{wrapfigure}
        

            \paragraph{} 
      


    \subsubsection{Reveal}
    
            \paragraph{}
            
            \begin{wrapfigure}{r}{0.25\textwidth}
                        \centering
                        \includegraphics[width=0.25\textwidth]{KiriZen/createUnfoldedPattern.png}
                        \caption{Icon}
                        \label{fig:kiriZen-createUnfoldedPattern}
                    \end{wrapfigure}

        \subsubsection{Transformation}
        
                \paragraph{}

    \subsection{Match Pattern}

    \subsubsection{Target Shapes}
    
            \paragraph{}
            \begin{figure}[!ht]
                        \begin{minipage}{0.45\textwidth}
                            \centering \includegraphics[width=0.7\linewidth]{KiriZen/simpleTarget.png}
                            \caption{aaaaa}
                            \label{fig:kiriZen-simpleTarget}
                        \end{minipage}\hfill
                        \begin{minipage}{0.45\textwidth}
                            \centering
                            \includegraphics[width=0.7\linewidth]{KiriZen/complexTarget.png}
                            \caption{aaaa}
                            \label{fig:kiriZen-complexTarget}
                        \end{minipage}
                    \end{figure}
                    
        \subsubsection{Inheritance}
    
            \paragraph{}
            \begin{figure}[!ht]
                        \begin{minipage}{0.45\textwidth}
                            \centering \includegraphics[width=0.7\linewidth]{KiriZen/overlayPopUp.png}
                            \caption{aaaaa}
                            \label{fig:kiriZen-overlayPopUp}
                        \end{minipage}\hfill
                        \begin{minipage}{0.45\textwidth}
                            \centering
                            \includegraphics[width=0.7\linewidth]{KiriZen/matchCreateOverlay.png}
                            \caption{aaaa}
                            \label{fig:kiriZen-matchCreateOverlay}
                        \end{minipage}
                    \end{figure}
            
            \paragraph{}
            
            \begin{wrapfigure}{r}{0.25\textwidth}
                        \centering
                        \includegraphics[width=0.25\textwidth]{KiriZen/matchUnfoldedPattern.png}
                        \caption{Icon}
                        \label{fig:kiriZen-matchUnfoldedPattern}
                    \end{wrapfigure}
      

\newpage
\section{Human Computer Interaction}
%heuiristics
    \paragraph{}
    
        \subsection{Evaluation}
            
                \paragraph{}

    \subsection{User Testing}
    
           \paragraph{} Throughout the project I asked my peers to test out the app and I observed how they interacted with it. For example I observed that other users had larger hands and fingers than mine so the application was harder to interact with. I addressed this issue by enlarging the button size, and the radius of initial circle. 
           
           In addition to this, I was informed by the users that the icons on the initial screen were very small when size by side so I made the decision to enlarge them and stack them vertically to see the icon image more clearly. 
           
           Undo and clear functionality - observe user interct with appp
                 After I incorporated this additional undo functionality I again observed the user interact with the application. 
           
           added lines after user feedback
           
           Instructions (started with help, then question mark, then setteled on the word - clearest)
           
           Share and save feature more important than initally expected.
        


    
    \subsection{Theme}
        
        \paragraph{}
        The name of the app is a combination of the words "kiri" and "zen". Kiri as previously mentioned referes to the Japanese word meaning to cut. The word Zen  associated with ..... is Japanese word meditation ........ 
        
        Upon researching the name no similar applications, products or concepts that could potentially lead users elsewhere were found. If I were to publish the application it would be easy for users to find as there would be no ambiguity regarding the ....
        
        
        share links to the application or find quickly 
        
        to find and share meaning there would be no conflicts.

        \subsubsection{Additional Features}
        \paragraph{}
        The additional features of the application, while not crucial to the core function, are crucial to creating an overarching theme for the application creating a more complete experience for the user.    Additional features .... e.g. music player, information page, icons, 
        
         \subsubsection{Graphics}
    
        \begin{wrapfigure}{r}{0.25\textwidth}
                        \centering
                        \includegraphics[width=0.25\textwidth]{KiriZen/icon.png}
                        \caption{Icon}
                        \label{fig:icon}
                    \end{wrapfigure}
            I created the icon in \textbf{Figure~\ref{fig:icon}} using the online tool canva %https://www.canva.com 
            to create the icon for the application. I chose a butterfly as they are symmetrical and the image
            
            I used the tool appicon %https://appicon.co
            to automatically generate the required images sized for different resolutions for different iPhone models. This is to allow the images to have the best possible resolution for the device it is being displayed on.  
            
            Need multiple sizes as the small components of the image – if downscaled from the large image would disappear when downscaled i.e. from 3x to 1x image. 
        
            I also used canva to create icons for all the buttons as  seen in fig x and y. I thought it was important to create the images from scratch rather than use stock images to make them personal to my app, so unlike some of the reviewed apps, the images are relevant and clearly relate to their function. I think it is also important to use original images to help create a uniqueness to my app, but in the case of the buttons on the main "creation" screens, it is more important that the functionality is clear than having an attractive icon which is why I generated the buttons from scratch with single words rather than inserting images into a button.
            
            I kept the colour palette consistent throughout and initially picking pastel colours for a calming modern  aura rather than saturated colours reminiscent from games from the 80s. I changed the navigation bar's colour to pink rather than the default blue to fit in with the colour scheme. 
            
            
            buttons - animations, curve, change overlay highlight. 
            
            \subsubsection{Share and Save}
            \begin{figure}[!ht]
                        \begin{minipage}{0.45\textwidth}
                            \centering \includegraphics[width=0.7\linewidth]{KiriZen/share.png}
                            \caption{aaaaa}
                            \label{fig:kiriZen-share}
                        \end{minipage}\hfill
                        \begin{minipage}{0.45\textwidth}
                            \centering
                            \includegraphics[width=0.7\linewidth]{KiriZen/main.png}
                            \caption{aaaaa}
                            \label{fig:kiriZen-main}
                        \end{minipage}
                    \end{figure}
            
             \subsubsection{Music Player}
                I have music playing in the background associated
                I added the music to enhance the mindfulness aspect of the application. As  I contacted the artist as the music was
                
                %cite music
                There is an on off toggle button. 
                
                %cite https://www.zerotoappstore.com/how-to-add-background-music-in-swift.html
                
        \subsection{Instructions}
            
                \paragraph{}
               \begin{figure}[!ht]
                                \begin{minipage}{0.45\textwidth}
                                    \centering \includegraphics[width=0.7\linewidth]{KiriZen/instructionsCreate.png}
                                    \caption{aaaaa}
                                    \label{fig:kiriZen-instructionsCreate}
                                \end{minipage}\hfill
                                \begin{minipage}{0.45\textwidth}
                                    \centering
                                    \includegraphics[width=0.7\linewidth]{KiriZen/instructionsMatch.png}
                                    \caption{aaaa}
                                    \label{fig:kiriZen-instructionsMatch}
                                \end{minipage}
                            \end{figure}



                 \subsubsection{Launch Screen}

                    \begin{figure}[!ht]
                        \begin{minipage}{0.45\textwidth}
                            \centering \includegraphics[width=0.7\linewidth]{KiriZen/launchPage.png}
                            \caption{aaaaa}
                            \label{fig:kiriZen-launch}
                        \end{minipage}\hfill
                        \begin{minipage}{0.45\textwidth}
                            \centering
                            \includegraphics[width=0.7\linewidth]{KiriZen/info.png}
                            \caption{aaaaa}
                            \label{fig:kiriZen-info}
                        \end{minipage}
                    \end{figure}
            
                I incorporated a loading screen while the application is starting up so the user can view a page that is relevant to the application rather than the default white page. This is especially important if the application is taking a long time to open - usually this happens on the initial load if the application is not already running in the background.
                

                
                 \subsubsection{Information Page}
                
                The information page serves as additional insight for the user if they would like some additional information about the application. 
                
               



    \subsection{Deployment}
        \paragraph{}
        put on app store
        could do seasonal marketing if near Christmas -e.g. extra snowflake designs and animations
        
        The code should be tested on the latest beta version of Xcode software as it will contain extra upcoming features and I will be able to test it on the updates that will be released shortly in the future. 
        
    
\newpage
\section{Discussion}
    \paragraph{}

    
    \subsection{Achievements}
    
        \paragraph{}
        Creating an application that unlike the other iPhone application has consistent, reliable and predictable cutting action. 
        
        I created an application where the majority of users reported they felt relaxed while using the application confirming that it can be used 
        as an effective part of the mindfulness practice. 
        in the context of as part of the practice of mindfulness 
    
    \subsection{Challenges}
    
        \paragraph{}
        How the user should cut. It was challenging as I was unfamiliar with how the iPhone collects data from user interactions and what can be done in real time. 
        
        
        Layering the shapes on the screen on top of dsashed line and cut.
        
        
        Reveal page transformations. 
        
        Limitation 
        compatability 
        
    \subsection{Architecture}
        
        \paragraph{}
using inheritance ... 
         flexibility, scalability, feasibility, reusability, and security into a structured solution
         Refactoring, common buttons, canvas, reveal. 
        
        \paragraph{}
        While creating making smaller methods
            
            
    \subsection{Future}
        
                \paragraph{}
                
                Animation of folds
                
                Higher number of folds - 10, 12...
                
                Match pattern section
                Additional functionality could be added to the app such as an algorithm to generate random target images shaped but more sophisticated than randomly rendering shapes - some restrictions could apply such as shapes not touching - the variate of shapes etc.
                
                Algorithms for comparing target images to the image the user created 
                
                
                Wallpaper
                
\newpage
\section{Conclusion}
        
            \paragraph{}

        
\newpage
\let\Section\section 
\def\section*#1{\Section{#1}}  
\bibliographystyle{IEEEtran}
    \bibliography{bibliography}
    
\newpage
\Section{Appendix}

        \begin{enumerate}
            \item On a Mac running macOS 10.14.3 or later
            \item Download Xcode
            \item Download the project from 
            
            https://git-teaching.cs.bham.ac.uk/mod-msc-proj-2018/sct808
            \item Open "kirigami draft 1.xcodeproj"
            \item Run on an iPhone 8 simmulator. 
            (I believe it is not possible to run on your own iPhone without already being an app developer.)
      
        \end{enumerate}
        
% \begin{figure}[h!]
% \centering
% \includegraphics[scale=1.7]{universe}
% \caption{caption}
% \label{fig:nameof file}
% \end{figure}

\end{document}
